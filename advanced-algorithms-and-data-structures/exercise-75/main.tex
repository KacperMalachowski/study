%----------------------- Wydruk dwustronny ---------------
%\documentclass[12pt,twoside,a4paper]{book} % 
%----------------------- Wydruk jednostronny ---------------
\documentclass[12pt,oneside,a4paper]{book} % jednostronnego

\usepackage{polski}
\usepackage[utf8]{inputenc} %opcja dla edytorów kodujących polskie znaki w utf8
%\usepackage[cp1250]{inputenc} %opcja dla edytorów kodujących polskie znaki w windows-1250
\usepackage{lmodern}
\usepackage{indentfirst}
\usepackage[protrusion=false]{microtype}
\DisableLigatures{encoding = *, family = * }
\usepackage{fancyhdr}
\usepackage{pstricks,graphicx}
\usepackage{amssymb}
\usepackage{float}

\usepackage{pdflscape}
\usepackage{diagbox}

%---------------Zbiory liczbowe
\newcommand{\R}{\mathbb{R}}
\newcommand{\N}{\mathbb{N}}
\newcommand{\K}{\mathbb{K}}
\newcommand{\C}{\mathcal{C}}
\newcommand{\p}{\mathcal{P}}
%------------kwantyfikatory--------------
\newcommand{\fal}{\mbox{{\Large $\forall\,$}}}
\newcommand{\ext}{\mbox{{\Large $\exists\,$}}}
%------------------definicje środowisk-----------------
\usepackage{theorem}
\theoremstyle{break}
\theorembodyfont{\it}
\newtheorem{twr}{Twierdzenie}[chapter]
\newtheorem{lem}{Lemat}[chapter]
\theorembodyfont{\rm}
\newtheorem{defi}{Definicja}[chapter]
\newtheorem{wni}{Wniosek}[chapter]
\newtheorem{prz}{Przykład}[chapter]
\newenvironment{dowod}{\par\vspace{0.1cm}\par{ \sc Dowód.}}{\hfill $\blacksquare$\par\vspace{0.4cm}\par}
% ----------ustawienia wymiarow strony
\usepackage{geometry}

\newgeometry{tmargin=2.5cm, bmargin=2.5cm, headheight=14.5pt, inner=3cm, outer=2.5cm} 

\linespread{1.1} %-zmiana interlinii


\fancypagestyle{mylandscape}{
\fancyhf{} %Clears the header/footer
\fancyfoot{% Footer
\makebox[\textwidth][r]{% Right
  \rlap{\hspace{.75cm}% Push out of margin by \footskip
    \smash{% Remove vertical height
      \raisebox{4.87in}{% Raise vertically
        \rotatebox{90}{\thepage}}}}}}% Rotate counter-clockwise
\renewcommand{\headrulewidth}{0pt}% No header rule
\renewcommand{\footrulewidth}{0pt}% No footer rule
}

%---------------- Normalne środowiska --------------------
\usepackage{amsmath}

%----------nagłowki i żywa pagina ------------
\pagestyle{fancy} 
%--------------- Wydruk dwustronny
%\cfoot[]{} 
%\lhead[{\scriptsize{\it \thepage}}]{}
%\chead[{\scriptsize\leftmark}]{{\scriptsize \rightmark}}
%\rhead[]{{\scriptsize{\it \thepage}}}
%--------------- Wydruk jednostronny
\fancyhead[C]{} 
\fancyfoot[C]{\thepage}
\fancyhead[L]{\scriptsize\leftmark}
\fancyhead[R]{\scriptsize\rightmark}

\renewcommand{\chaptermark}[1]{%
\markboth{\MakeUppercase{%
\chaptername}\ \thechapter.%
\ #1}{}}

\usepackage[most]{tcolorbox}
\let\includegraphicsold\includegraphics
\newcommand{\includegraphicsborder}[2][]{\tcbox{\includegraphicsold[#1]{#2}}}

\renewcommand{\sectionmark}[1]{\markright{\thesection.\ #1}}

\usepackage[hidelinks]{hyperref}

\usepackage{graphics}
\graphicspath{ {images/} }

\usepackage{listings}

\renewcommand{\lstlistlistingname}{Spis listingów}
\renewcommand{\lstlistingname}{Listing}

\lstset{
  basicstyle=\footnotesize,
  numbers=left,
  mathescape=true
}

\usepackage{booktabs}

\newcommand\tabularhead[2]{
  \begin{table}[ht]
    \label{#2}
    \caption{#1}
    \begin{tabular}{|p{0.35\linewidth}|p{0.6\linewidth}|}
    \hline
    \textbf{#1}\\
    \hline
}
\newcommand\addrow[2]{#1 &#2\\ \hline}

\newcommand\addmulrow[2]{ \begin{minipage}[t][][t]{2.5cm}#1\end{minipage}
   &\begin{minipage}[t][][t]{8cm}
    \begin{enumerate} #2   \end{enumerate}
    \end{minipage}\\ }

\newenvironment{usecase}{\tabularhead}
{\hline\end{tabular}\end{table}}



%-----------------właściwa część pracy-----------------
\begin{document}
\thispagestyle{empty}
\begin{center}
  \Large
  \bf{UNIWERSYTET ŚLĄSKI}\\
  \bf{\sf{WYDZIAŁ NAUK ŚCISŁYCH I TECHNICZNYCH}}\\[25mm]
  \large

  \bf{Zadanie 75 - Rzuty monetą}\\[35mm]

  Sprawozdanie\\
  z przedmiotu\\
  Zaawansowane algorytmy i struktury danych\\[25mm]
\end{center}
\begin{flushright}
  \large
  Autorzy:\\
  Kacper Małachowski\\
\end{flushright}
\vspace*{\fill}
\begin{center}
  Informatyka II Stopnia\\
  Lato 2023/2024\\
  I rok, grupa 3\\[25mm]
\end{center}

\chapter*{Treść Zadania 75 - Rzuty Monetą}

Załóżmy, że monetą rzucamy n razy. Przez ciąg powtórzeń rozumiemy kolejne po sobie rzuty, w których ta sama strona monety (za każdym razem orzeł lub za każdym razem reszka). Na przykład w OOORRORO (n = 8) występuje 5 ciągów powtórzeń.

Jaka jest spodziewana liczba ciągów powtórzeń?

\chapter*{Teoria}

Rozważmy proces rzutów monetą. Przestrzeń $\Omega$ dla pojedynczego rzutu monetą wynosi $\Omega=\{O, R\}$, gdzie $O$ - orzeł oraz $R$ - reszka. Zakładamy, że rzucamy uczciwą monetą, dlatego za każdym razem prawdopodobieństwo wyrzucenia orła $P(O)$ jest równe prawdopodobieństwu wyrzuceniu reszki $P(R)$. Z uwagi na istnienie tylko dwóch możliwych wyników losowania, nie rozpatrujemy innego (np. pionowego) ułożenia monety, możemy określić prawdopodobieństwo obu przypadków $P(O)=P(R)=\frac{1}{2}$.
Zdarzenia są niezależne gdy $P(A \cap B) = P(A)$ \cite{MetProb}.

Proces rzutów monetą jest częstym przykładem stosowanym do prezentacji klasycznego rachunku prawdopodobieństwa, głównie przez swoją prostote w odniesieniu do namacalnego eksperymentu, który każdy moze przeprowadzić na żywo.

Wartością oczekiwaną zmiennej losowej X nazywamy wartość sumy
\[
  E(X)=\sum_{k}P(X=k)*k
\]

gdzie przez k oznaczono możliwe wartości dla zmiennej X \cite{OczekiwanaMIMUW}.




\chapter*{Rozwiązanie}

Jako $X_i$ oznaczmy zmienną, która przyjmuje wartość 1 jeśli rzut rozpoczyna nowy ciąg i 0 w przeciwnym wypadku. Ponieważ pierwszy rzut rozpoczyna nowy ciąg powtórzeń, a więc $P(X_1=1)=1$ to rozważmy rzuty od 2 do n. 

Korzystając zatem ze wzoru na wartosć oczekiwaną zmiennej losowej otrzymujemy:

\[
  E(liczbaCiagow) = \sum_{i=2}^{n}E(X_i)
\]

Ponieważ kolejne rzuty są niezależne, prawdopodobieństwo rozpoczęcia nowego ciągu wynosi $P(X_i=1)=\frac{1}{2}$ dla każdego rzutu. Stąd: \[E(X_i)=P(X_i=1)*1+P(X_i=0)*0=P(X_i=1)=\frac{1}{2}\]. Teraz obliczmy sumę wszystkich $X_i$ od i = 2:

\[
  E(liczbaCiagow)=\sum_{i=2}^{n}E(X_i) = \sum_{i=2}^{n}\frac{1}{2} = (n-1)*\frac{1}{2} = \frac{n-1}{2}
\]

Ponieważ pierwszy rzut zawsze rozpoczyna nowy ciąg to do tak uzyskanego wzoru dodajemy 1, stąd:

\[
  E(liczbaCiagow) = 1 + \frac{n-1}{2} = \frac{n+1}{2}
\]

Zatem odpowiedź to: Spodziewana liczba ciagów powtórzeń wynosi $\frac{n+1}{2}$.

\begin{thebibliography}{00}
  \bibitem{MetProb}
  Kołtowski, W. (2020, November 4). Metody probabilistyczne: 4. Niezależność. Instytut Informatyki PP.\\ \url{https://www.cs.put.poznan.pl/wkotlowski/mp/04\_niezaleznosc.pdf}

  \bibitem{OczekiwanaMIMUW}
  Uniwersytet Warszawski. Wartość oczekiwana\\
  \url{http://smurf.mimuw.edu.pl/uczesie/?q=node/181}

\end{thebibliography}
\end{document}