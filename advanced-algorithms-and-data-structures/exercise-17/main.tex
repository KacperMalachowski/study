%----------------------- Wydruk dwustronny ---------------
%\documentclass[12pt,twoside,a4paper]{book} % 
%----------------------- Wydruk jednostronny ---------------
\documentclass[12pt,oneside,a4paper]{book} % jednostronnego

\usepackage{polski}
\usepackage[utf8]{inputenc} %opcja dla edytorów kodujących polskie znaki w utf8
%\usepackage[cp1250]{inputenc} %opcja dla edytorów kodujących polskie znaki w windows-1250
\usepackage{lmodern}
\usepackage{indentfirst}
\usepackage[protrusion=false]{microtype}
\DisableLigatures{encoding = *, family = * }
\usepackage{fancyhdr}
\usepackage{pstricks,graphicx}
\usepackage{amssymb}
\usepackage{float}

\usepackage{pdflscape}
\usepackage{diagbox}

%---------------Zbiory liczbowe
\newcommand{\R}{\mathbb{R}}
\newcommand{\N}{\mathbb{N}}
\newcommand{\K}{\mathbb{K}}
\newcommand{\C}{\mathcal{C}}
\newcommand{\p}{\mathcal{P}}
%------------kwantyfikatory--------------
\newcommand{\fal}{\mbox{{\Large $\forall\,$}}}
\newcommand{\ext}{\mbox{{\Large $\exists\,$}}}
%------------------definicje środowisk-----------------
\usepackage{theorem}
\theoremstyle{break}
\theorembodyfont{\it}
\newtheorem{twr}{Twierdzenie}[chapter]
\newtheorem{lem}{Lemat}[chapter]
\theorembodyfont{\rm}
\newtheorem{defi}{Definicja}[chapter]
\newtheorem{wni}{Wniosek}[chapter]
\newtheorem{prz}{Przykład}[chapter]
\newenvironment{dowod}{\par\vspace{0.1cm}\par{ \sc Dowód.}}{\hfill $\blacksquare$\par\vspace{0.4cm}\par}
% ----------ustawienia wymiarow strony
\usepackage{geometry}

\newgeometry{tmargin=2.5cm, bmargin=2.5cm, headheight=14.5pt, inner=3cm, outer=2.5cm} 

\linespread{1.1} %-zmiana interlinii


\fancypagestyle{mylandscape}{
\fancyhf{} %Clears the header/footer
\fancyfoot{% Footer
\makebox[\textwidth][r]{% Right
  \rlap{\hspace{.75cm}% Push out of margin by \footskip
    \smash{% Remove vertical height
      \raisebox{4.87in}{% Raise vertically
        \rotatebox{90}{\thepage}}}}}}% Rotate counter-clockwise
\renewcommand{\headrulewidth}{0pt}% No header rule
\renewcommand{\footrulewidth}{0pt}% No footer rule
}

%---------------- Normalne środowiska --------------------
\usepackage{amsmath}

%----------nagłowki i żywa pagina ------------
\pagestyle{fancy} 
%--------------- Wydruk dwustronny
%\cfoot[]{} 
%\lhead[{\scriptsize{\it \thepage}}]{}
%\chead[{\scriptsize\leftmark}]{{\scriptsize \rightmark}}
%\rhead[]{{\scriptsize{\it \thepage}}}
%--------------- Wydruk jednostronny
\fancyhead[C]{} 
\fancyfoot[C]{\thepage}
\fancyhead[L]{\scriptsize\leftmark}
\fancyhead[R]{\scriptsize\rightmark}

\renewcommand{\chaptermark}[1]{%
\markboth{\MakeUppercase{%
\chaptername}\ \thechapter.%
\ #1}{}}

\usepackage[most]{tcolorbox}
\let\includegraphicsold\includegraphics
\newcommand{\includegraphicsborder}[2][]{\tcbox{\includegraphicsold[#1]{#2}}}

\renewcommand{\sectionmark}[1]{\markright{\thesection.\ #1}}

\usepackage[hidelinks]{hyperref}

\usepackage{graphics}
\graphicspath{ {images/} }

\usepackage{listings}

\renewcommand{\lstlistlistingname}{Spis listingów}
\renewcommand{\lstlistingname}{Listing}

\lstset{
  basicstyle=\footnotesize,
  numbers=left,
  mathescape=true
}

\usepackage{booktabs}

\newcommand\tabularhead[2]{
  \begin{table}[ht]
    \label{#2}
    \caption{#1}
    \begin{tabular}{|p{0.35\linewidth}|p{0.6\linewidth}|}
    \hline
    \textbf{#1}\\
    \hline
}
\newcommand\addrow[2]{#1 &#2\\ \hline}

\newcommand\addmulrow[2]{ \begin{minipage}[t][][t]{2.5cm}#1\end{minipage}
   &\begin{minipage}[t][][t]{8cm}
    \begin{enumerate} #2   \end{enumerate}
    \end{minipage}\\ }

\newenvironment{usecase}{\tabularhead}
{\hline\end{tabular}\end{table}}



%-----------------właściwa część pracy-----------------
\begin{document}
\thispagestyle{empty}
\begin{center}
  \Large
  \bf{UNIWERSYTET ŚLĄSKI}\\
  \bf{\sf{WYDZIAŁ NAUK ŚCISŁYCH I TECHNICZNYCH}}\\[25mm]
  \large

  \bf{Zadanie 17 - Błądzenie losowe}\\[35mm]

  Sprawozdanie\\
  z przedmiotu\\
  Zaawansowane algorytmy i struktury danych\\[25mm]
\end{center}
\begin{flushright}
  \large
  Autorzy:\\
  Kacper Małachowski\\
\end{flushright}
\vspace*{\fill}
\begin{center}
  Informatyka II Stopnia\\
  Lato 2023/2024\\
  I rok, grupa 3\\[25mm]
\end{center}

\chapter*{Treść Zadania 17 - Błądzenie losowe}

Rozważmy poniższy algorytm, w którym funkcja random zwraca liczbę losową z przedziału [0,1).
\lstset{morekeywords={while, do, if, then, else}}
\begin{lstlisting}
  RandomWalks(n)
    i = 1
    while i < n do
      r := random()
      if (r $\ge \frac{1}{2}$) $\lor$ (i == 1) then
        i = i + 1
      else i = i - 1
\end{lstlisting}

Wykaż, że wartość oczekiwania liczby kroków tego algorytmu, po wykonaniu których następuje jego zakończenie (zmienna \textit{i} przyjmuje wartość \textit{n}) jest rzędu \textit{$O(n^2)$}

\chapter*{Teoria}

Błądzenie losowe (ang. \textit{Random Walk}) - jest procesem, w którym obiekt wędruje w sposób losowy od miejsca startu \cite{viriginaRandomWalk}.

Algorytm - ciąg instrukcji wymaganych do rozwiązania konkretnego zadania.

Prawdopodobieństwo - szansa na zaistnienie pewnego zdarzenia.

\chapter*{Rozwiązanie}

Zadaniem jest dowieść, że oczekiwana liczba kroków dla tego algorytmu do jego zakończenia jest rzędu $O(n^2)$. Zauważyć należy najpierw, że mamy do czynienia z jednowymiarowym błądzeniem losowym. Co prawda w omawianym przypadku jest ono ograniczone ze lewej strony przez warunek, który uniemożliwia przejście na wartości ujemne, jednak dla potrzeb oszacowania ilości kroków nie ma to znaczenia z uwagi na losowość pozostałych kroków i naturę O-notacji, czyli oszacowania górnego.

Zauważyć również należy, że tak skonstruowany warunek powoduje, że prawdopodobieństwo pójścia do przodu ($n_f$) i pójścia do tyłu ($n_b$) są takie same. Dzieje się tak ponieważ prawdopodobieństwo wylosowanie wartości większej lub równej $\frac{1}{2}$ jest równa prawdopodobieństwo wylosowania wartości mniejszej od tej wartości. Co można zapisać jako $P(n_f)=P(n_b)=\frac{1}{2}$.

Z analizy algorytmu widzimy, że parametr n jest w rzeczywistości oczekiwanym dystansem, który algorytm ma pokonać zanim się zakończy.

Spróbujmy zatem policzyć średni dystans, który pokona algorytm po N krokach. Wiemy, że dla każdego kroku poza pierwszym (ponieważ i = 1) jest równe dla każdego kroku, stąd $\overline{a_i}=0$ dla $i > 1$.

\begin{math}
  \centering
  \overline{n}= \overline{a_1}+\overline{a_2}+\overline{a_3}+...+\overline{a_n}=1+0+0+...+0=1
\end{math}

Oznacza, to że średnio po N krokach algorytm przesunie się o jedną pozycje. To prowadzi do wniosku, że taki algorytm nie zakończy się nigdy dla $n > 1$, co przeczy teorii błądzenia losowego \cite{viriginaRandomWalk}.

Spróbujmy zatem policzyć $\overline{n^2}$, ponieważ o ile średnia wartosć dla każdego kroku poza pierwszym wynosi 0, to średni kwadrat tej wartości wynosi 1. Stąd:

\begin{math}
  \centering
  \overline{n^2} = \overline{(a_1 + a_2 + a_3 + ... + a_n)^2} = \overline{(a_1 + a_2 + a_3 + ... + a_n)(a_1 + a_2 + a_3 + ... + a_n)} = (\overline{a_1^2} + \overline{a_2^2} + ... + \overline{a_n^2}) + 2(\overline{a_1a_2} + \overline{a_1a_3} + ... + \overline{a_{n-1}a_n}) = (1 + 1 + 1 + ... + 1) + 2(0 + 0 + 0 + ... + 0) = N
\end{math}

Z tej równości możemy pokazać, że $\overline{n^2} = N$, to zaś dowodzi że dla dystansu $n$, potrzebujemy $N^2$ kroków by go osiągnąć w omawianym algorytmie i tym samym zakończyć ten algorytm. To zaś zgadza się z ogólną zasadą dla błądzenia losowego.

\begin{thebibliography}{00}
  \bibitem{viriginaRandomWalk}
  Michael Fowler. "The One-Dimensional Random Walk". \url{https://galileo.phys.virginia.edu/classes/152.mf1i.spring02/RandomWalk.htm}
\end{thebibliography}

\end{document}