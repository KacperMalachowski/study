%----------------------- Wydruk dwustronny ---------------
%\documentclass[12pt,twoside,a4paper]{book} % 
%----------------------- Wydruk jednostronny ---------------
\documentclass[12pt,oneside,a4paper]{book} % jednostronnego

\usepackage{polski}
\usepackage[utf8]{inputenc} %opcja dla edytorów kodujących polskie znaki w utf8
%\usepackage[cp1250]{inputenc} %opcja dla edytorów kodujących polskie znaki w windows-1250
\usepackage{lmodern}
\usepackage{indentfirst}
\usepackage[protrusion=false]{microtype}
\DisableLigatures{encoding = *, family = * }
\usepackage{fancyhdr}
\usepackage{pstricks,graphicx}
\usepackage{amssymb}
\usepackage{float}

\usepackage{pdflscape}
\usepackage{diagbox}

%---------------Zbiory liczbowe
\newcommand{\R}{\mathbb{R}}
\newcommand{\N}{\mathbb{N}}
\newcommand{\K}{\mathbb{K}}
\newcommand{\C}{\mathcal{C}}
\newcommand{\p}{\mathcal{P}}
%------------kwantyfikatory--------------
\newcommand{\fal}{\mbox{{\Large $\forall\,$}}}
\newcommand{\ext}{\mbox{{\Large $\exists\,$}}}
%------------------definicje środowisk-----------------
\usepackage{theorem}
\theoremstyle{break}
\theorembodyfont{\it}
\newtheorem{twr}{Twierdzenie}[chapter]
\newtheorem{lem}{Lemat}[chapter]
\theorembodyfont{\rm}
\newtheorem{defi}{Definicja}[chapter]
\newtheorem{wni}{Wniosek}[chapter]
\newtheorem{prz}{Przykład}[chapter]
\newenvironment{dowod}{\par\vspace{0.1cm}\par{ \sc Dowód.}}{\hfill $\blacksquare$\par\vspace{0.4cm}\par}
% ----------ustawienia wymiarow strony
\usepackage{geometry}

\newgeometry{tmargin=2.5cm, bmargin=2.5cm, headheight=14.5pt, inner=3cm, outer=2.5cm} 

\linespread{1.1} %-zmiana interlinii


\fancypagestyle{mylandscape}{
\fancyhf{} %Clears the header/footer
\fancyfoot{% Footer
\makebox[\textwidth][r]{% Right
  \rlap{\hspace{.75cm}% Push out of margin by \footskip
    \smash{% Remove vertical height
      \raisebox{4.87in}{% Raise vertically
        \rotatebox{90}{\thepage}}}}}}% Rotate counter-clockwise
\renewcommand{\headrulewidth}{0pt}% No header rule
\renewcommand{\footrulewidth}{0pt}% No footer rule
}

%---------------- Normalne środowiska --------------------
\usepackage{amsmath}

%----------nagłowki i żywa pagina ------------
\pagestyle{fancy} 
%--------------- Wydruk dwustronny
%\cfoot[]{} 
%\lhead[{\scriptsize{\it \thepage}}]{}
%\chead[{\scriptsize\leftmark}]{{\scriptsize \rightmark}}
%\rhead[]{{\scriptsize{\it \thepage}}}
%--------------- Wydruk jednostronny
\fancyhead[C]{} 
\fancyfoot[C]{\thepage}
\fancyhead[L]{\scriptsize\leftmark}
\fancyhead[R]{\scriptsize\rightmark}

\renewcommand{\chaptermark}[1]{%
\markboth{\MakeUppercase{%
\chaptername}\ \thechapter.%
\ #1}{}}

\usepackage[most]{tcolorbox}
\let\includegraphicsold\includegraphics
\newcommand{\includegraphicsborder}[2][]{\tcbox{\includegraphicsold[#1]{#2}}}

\renewcommand{\sectionmark}[1]{\markright{\thesection.\ #1}}

\usepackage[hidelinks]{hyperref}

\usepackage{graphics}
\graphicspath{ {images/} }

\usepackage{listings}

\renewcommand{\lstlistlistingname}{Spis listingów}
\renewcommand{\lstlistingname}{Listing}

\lstset{
  basicstyle=\footnotesize,
  numbers=left,
  mathescape=true
}

\usepackage{booktabs}

\newcommand\tabularhead[2]{
  \begin{table}[ht]
    \label{#2}
    \caption{#1}
    \begin{tabular}{|p{0.35\linewidth}|p{0.6\linewidth}|}
    \hline
    \textbf{#1}\\
    \hline
}
\newcommand\addrow[2]{#1 &#2\\ \hline}

\newcommand\addmulrow[2]{ \begin{minipage}[t][][t]{2.5cm}#1\end{minipage}
   &\begin{minipage}[t][][t]{8cm}
    \begin{enumerate} #2   \end{enumerate}
    \end{minipage}\\ }

\newenvironment{usecase}{\tabularhead}
{\hline\end{tabular}\end{table}}



%-----------------właściwa część pracy-----------------
\begin{document}
\thispagestyle{empty}
\begin{center}
  \Large
  \bf{UNIWERSYTET ŚLĄSKI}\\
  \bf{\sf{WYDZIAŁ NAUK ŚCISŁYCH I TECHNICZNYCH}}\\[25mm]
  \large

  \bf{Zadanie 17 - Błądzenie losowe}\\[35mm]

  Sprawozdanie\\
  z przedmiotu\\
  Zaawansowane algorytmy i struktury danych\\[25mm]
\end{center}
\begin{flushright}
  \large
  Autorzy:\\
  Kacper Małachowski\\
\end{flushright}
\vspace*{\fill}
\begin{center}
  Informatyka II Stopnia\\
  Lato 2023/2024\\
  I rok, grupa 3\\[25mm]
\end{center}

\chapter*{Treść Zadania 17 - Błądzenie losowe}

Rozważmy poniższy algorytm, w którym funkcja random zwraca liczbę losową z przedziału [0,1).
\lstset{morekeywords={while, do, if, then, else}}
\begin{lstlisting}
  RandomWalks(n)
    i = 1
    while i < n do
      r := random()
      if (r $\ge \frac{1}{2}$) $\lor$ (i == 1) then
        i = i + 1
      else i = i - 1
\end{lstlisting}

Wykaż, że wartość oczekiwania liczby kroków tego algorytmu, po wykonaniu których następuje jego zakończenie (zmienna \textit{i} przyjmuje wartość \textit{n}) jest rzędu \textit{$O(n^2)$}

\chapter*{Teoria}

Jedną z metod rozwiązania równania rekurencyjnego, co pozowoli nam wykazać, że wymagana liczba kroków jest rzędu $O(n^2)$ jest metoda równania charaktetystycznego.

Jak pokazano w rozwiązaniu, mamy do czynienia z rekurencją niejednorodną, zatem rozwiązanie ma postać $a_n=a_n^h+a_n^p$, gdzie $a_n^h$ to rozwiązanie ogólne równania jednorodnego, natomiast $a_n^p$ to rozwiązanie szczegółowe równania niejednorodnego \cite{KwasnyRekurencja}.

Równanie charakterystyczne stosuje się do rekurencji w postaci:
\begin{align*}
  \begin{cases}
    a_0,a_1\\
    a_n=c_1a_{n-1}=c_0a_{n-2}
  \end{cases}
\end{align*}
gdzie $c_1$ oraz $c_0$ są pewnymi stałymi, wtedy równanie to ma postać:
\begin{align*}
  x^2-c_1x-c_0=0
\end{align*}
co jest prostsze do rozwiązania \cite{BorowskaAlgorytmy}.

Ponieważ równanie charakterystyczne jest w tym przypadku równaniem kwadratowym, należy obliczyć jego pierwiastki, stąd przypomnienie równania na deltę:
\begin{align*}
  \Delta=b^2-4ac
\end{align*}

i pierwiastki równania kwadratowe dla $\Delta > 0$:
\begin{align*}
  x_1=\frac{-b-\sqrt{\Delta}}{2a}\\
  x_2=\frac{-b+\sqrt{\Delta}}{2a}
\end{align*}
oraz dla $\Delta=0$:
\begin{align*}
  x=\frac{-b}{2a}
\end{align*}


\chapter*{Rozwiązanie}

W celu rozwiązania podanego zadania, zauważyć należy, że mamy do czynienia w istocie z funkcją rekurencyjną. Co również ważne, zauważyć należe, że przez konstrukcje warunku w linii 5, prawdopodbieństwo pójścia do przodu $P(f)$ jest równe prawdopodbieństwu pójścia do tyłu $P(b)$. Można to zapisać jako $P(f)=P(b)=\frac{1}{2}$ dla $i > 1$.

Ponadto mamy do czynienia z algorytmem rekurencyjnym z odbiciem w punkcie 1 oraz absorcją w definiowanym przez użytkownika punkcie n. W związku z tym prawdopodbieństwo przejścia do drugiego kroku dla $i=1$ wynosi 1.

Możemy zatem określić warunki początkowe wraz z wzorem ogólnym rekurencji. Ponieważ jak wspomniano dla $E_n$ istnieje bariera absorbująca to wartość rekurencji jest równa 0.
Natomiast z powodu odbicia na pozycji 1, otrzymano $E_1=1+E_2$. Stąd zapis rekurencji w postaci:
\begin{align*}
  \begin{cases}
    E_n=0\\
    E_1=1+E_2\\
    E_i=1+\frac{1}{2}E_{i-1}+\frac{1}{2}E_{i+1}
  \end{cases}
\end{align*}

Zauważyć należy, że $E_i=1+\frac{1}{2}E_{i-1}+\frac{1}{2}E_{i+1}$ można uprościć do postaci $E_{i+1}=2E_i-E_{i-1}-2$.

Zatem, mamy do czynienia z dwiema częściami, jednorodną w postaci $E_{i+1}=2E_i-E{i-1}$ oraz niejednorodną $-2$. Zatem wzór charaktetystyczny dla postaci jednorodnej ma postać: 
\begin{align*}
  x^2-2x+1=0
\end{align*}
rozwiązaniem takiego równania jest:
\begin{align*}
  \Delta=b^2-4ac=4-4*1*1=4-4=0
\end{align*}
zatem równanie ma tylko jedno rozwiązanie:
\begin{align*}
  x=\frac{-b}{2a}=\frac{2}{2*1}=1
\end{align*}
stąd, rozwiązanie ogólne:
\begin{align*}
  a_n^h=A*r^i+B*n*r^1=A*1^n+B*n*r^1=A+Bn
\end{align*}

Znalezienie rozwiązania szczegółowego cześci niejednorodnej, wymaga opracowania wzoru nalepiej odzwieciedlającego rekurencje, w przypadku powyższego zadania będzie to $E_i^p=c+di=ei^2$. Podstawiając do $E_{i+1}=2E_i-E_{i-1}-2$, obliczamy.

\begin{align*}
  E_i=c+di+ei^2
\end{align*}
\begin{align*}
  E_{i-1}=c+d(i-1)+e(i-1)^2\\=c+d(i-1)+e(i^2-2i+1)\\=c+di-d+ei^2-2ei+e
\end{align*}
\begin{align*}
  E_{i+1}=c+d(i+1)+e(i+1)^2\\=c+d(i+1)+e(i^2+2i+1)\\=c+di+d+ei^2+2ei+e
\end{align*}
podstawmy do $E_{i+1}=2E_i-E_{i-1}-2$, rozwiażmy e:
\begin{align*}
  c+di+d+ei^2+2ei+e=2(c+di+ei^2)-(c+di-d+ei^2-2ei+e)-2\\
  c+di+d+ei^2+2ei+e=2c+2di+2ei^2-c-di+d-ei^2+2ei-e-2\\
  ei^2-2ei^2+ei^2+2ei-2ei+e+e=2c-c-c+2di-di-di+d-d-2\\
  2e=-2\\
  e=-1
\end{align*}

Na podstawie powyższych obliczeń widzimy, że $e=-1, c=0, d=0$.
Podstawiając do $E_i=c+di+ei^2$:
\begin{align*}
  E_i^p=0+0i-1i^2\\
  E_i^p=-i^2
\end{align*}

Zatem korzystając ze wzoru $E_i=E_i^h+E_i^p$, otrzymujemy $E_i=A+Bi-i^2$.
Obliczmy B, dla $E_n$. Korzystajac z $E_n=0$:
\begin{align*}
  0=A+Bi-i^2\\
  i^2=A+Bn
\end{align*}
Korzystajac z $E_1=1+E_2$:
\begin{align*}
  A+B-1^2=1+A+2B-2^2\\
  A+B-1=1+A+2B-4\\
  B-2B=1+1-4+A-A\\
  -B=-2\\
  B=2
\end{align*}
Podstawmy do wzoru $A+Bn=n^2$:
\begin{align*}
  A-2n=n^2\\
  A=n^2+2n
\end{align*}

Na koniec podstawiamy do $E_i=A+Bi-i^2$
\begin{align*}
  E_i=n^2+2n+2i-i^2
\end{align*}

Zatem spodziewana kroków wynosi $n^2+2n+2i-i^2$ co daje $O(n^2)$.
To zaś dowodzi, że spodziewana liczba kroków jest rzędu $O(n^2)$

\begin{thebibliography}{00}
  \bibitem{BorowskaAlgorytmy}
  Borowska, A. (2020). Algorytmy i struktury danych - ćwiczenia. Część I. Analiza i techniki projektowania algorytmów. Wydawnictwo Naukowe PWN.
  \bibitem{KwasnyRekurencja}
  Kwasny, J. Rozwiązywanie zależności rekurencyjnych metodą
  równania charakterystycznego. \url{https://home.agh.edu.pl/~jkwasny/WMD2018/rekurencja.pdf}\\
  Dostęp: 02.06.2024
\end{thebibliography}

\end{document}