%----------------------- Wydruk dwustronny ---------------
%\documentclass[12pt,twoside,a4paper]{book} % 
%----------------------- Wydruk jednostronny ---------------
\documentclass[12pt,oneside,a4paper]{book} % jednostronnego

\usepackage{polski}
\usepackage[utf8]{inputenc} %opcja dla edytorów kodujących polskie znaki w utf8
%\usepackage[cp1250]{inputenc} %opcja dla edytorów kodujących polskie znaki w windows-1250
\usepackage{lmodern}
\usepackage{indentfirst}
\usepackage[protrusion=false]{microtype}
\DisableLigatures{encoding = *, family = * }
\usepackage{fancyhdr}
\usepackage{pstricks,graphicx}
\usepackage{amssymb}
\usepackage{float}

\usepackage{pdflscape}
\usepackage{diagbox}

%---------------Zbiory liczbowe
\newcommand{\R}{\mathbb{R}}
\newcommand{\N}{\mathbb{N}}
\newcommand{\K}{\mathbb{K}}
\newcommand{\C}{\mathcal{C}}
\newcommand{\p}{\mathcal{P}}
%------------kwantyfikatory--------------
\newcommand{\fal}{\mbox{{\Large $\forall\,$}}}
\newcommand{\ext}{\mbox{{\Large $\exists\,$}}}
%------------------definicje środowisk-----------------
\usepackage{theorem}
\theoremstyle{break}
\theorembodyfont{\it}
\newtheorem{twr}{Twierdzenie}[chapter]
\newtheorem{lem}{Lemat}[chapter]
\theorembodyfont{\rm}
\newtheorem{defi}{Definicja}[chapter]
\newtheorem{wni}{Wniosek}[chapter]
\newtheorem{prz}{Przykład}[chapter]
\newenvironment{dowod}{\par\vspace{0.1cm}\par{ \sc Dowód.}}{\hfill $\blacksquare$\par\vspace{0.4cm}\par}
% ----------ustawienia wymiarow strony
\usepackage{geometry}

\newgeometry{tmargin=2.5cm, bmargin=2.5cm, headheight=14.5pt, inner=3cm, outer=2.5cm} 

\linespread{1.1} %-zmiana interlinii


\fancypagestyle{mylandscape}{
\fancyhf{} %Clears the header/footer
\fancyfoot{% Footer
\makebox[\textwidth][r]{% Right
  \rlap{\hspace{.75cm}% Push out of margin by \footskip
    \smash{% Remove vertical height
      \raisebox{4.87in}{% Raise vertically
        \rotatebox{90}{\thepage}}}}}}% Rotate counter-clockwise
\renewcommand{\headrulewidth}{0pt}% No header rule
\renewcommand{\footrulewidth}{0pt}% No footer rule
}

%---------------- Normalne środowiska --------------------
\usepackage{amsmath}

%----------nagłowki i żywa pagina ------------
\pagestyle{fancy} 
%--------------- Wydruk dwustronny
%\cfoot[]{} 
%\lhead[{\scriptsize{\it \thepage}}]{}
%\chead[{\scriptsize\leftmark}]{{\scriptsize \rightmark}}
%\rhead[]{{\scriptsize{\it \thepage}}}
%--------------- Wydruk jednostronny
\fancyhead[C]{} 
\fancyfoot[C]{\thepage}
\fancyhead[L]{\scriptsize\leftmark}
\fancyhead[R]{\scriptsize\rightmark}

\renewcommand{\chaptermark}[1]{%
\markboth{\MakeUppercase{%
\chaptername}\ \thechapter.%
\ #1}{}}

\usepackage[most]{tcolorbox}
\let\includegraphicsold\includegraphics
\newcommand{\includegraphicsborder}[2][]{\tcbox{\includegraphicsold[#1]{#2}}}

\renewcommand{\sectionmark}[1]{\markright{\thesection.\ #1}}

\usepackage[hidelinks]{hyperref}

\usepackage{graphics}
\graphicspath{ {images/} }

\usepackage{listings}

\renewcommand{\lstlistlistingname}{Spis listingów}
\renewcommand{\lstlistingname}{Listing}

\lstset{
  basicstyle=\footnotesize
}

\usepackage{booktabs}

\newcommand\tabularhead[2]{
  \begin{table}[ht]
    \label{#2}
    \caption{#1}
    \begin{tabular}{|p{0.35\linewidth}|p{0.6\linewidth}|}
    \hline
    \textbf{#1}\\
    \hline
}
\newcommand\addrow[2]{#1 &#2\\ \hline}

\newcommand\addmulrow[2]{ \begin{minipage}[t][][t]{2.5cm}#1\end{minipage}
   &\begin{minipage}[t][][t]{8cm}
    \begin{enumerate} #2   \end{enumerate}
    \end{minipage}\\ }

\newenvironment{usecase}{\tabularhead}
{\hline\end{tabular}\end{table}}



%-----------------właściwa część pracy-----------------
\begin{document}
\thispagestyle{empty}
\begin{center}
  \Large
  \bf{UNIWERSYTET ŚLĄSKI}\\
  \bf{\sf{WYDZIAŁ NAUK ŚCISŁYCH I TECHNICZNYCH}}\\[25mm]
  \large

  \bf{Porównanie klasyfikatorów z RSES}\\[35mm]

  Sprawozdanie\\
  z przedmiotu\\
  Eksploracja Danych\\[25mm]
\end{center}
\begin{flushright}
  \large
  Autorzy:\\
  Kacper Małachowski\\
\end{flushright}
\vspace*{\fill}
\begin{center}
  Informatyka II Stopnia\\
  Lato 2023/2024\\
  I rok, grupa 3\\[25mm]
\end{center}

\chapter*{Źródło danych}

Dane pochodzą z repozytorium uniwersytetu kalifornijskiego, gdzie znajdują się pod nazwą "Room Occupancy Estimation".
Stworzone zostały w ramach badań: Adarsh Pal Singh, Vivek Jain, Sachin Chaudhari, Frank Alexander Kraemer, Stefan Werner and Vishal Garg, "Machine Learning-Based Occupancy Estimation Using Multivariate Sensor Nodes," in 2018 IEEE Globecom Workshops (GC Wkshps), 2018.

\subsection*{Opis atrybutów}

\begin{itemize}
  \item Date - Pominięty w eksploracji - Data pomiaru z czujników.
  \item Time - Pominięty w eksploracji - Czas pomiaru z czujników z dokładnością do sekund.
  \item S1\_Temp - Odczyt z czujnika temperatury umieszczonego przy biurku nr 1, podany w stopniach celsjusza.
  \item S2\_Temp - Odczyt z czujnika temperatury umieszczonego przy biurku nr 2, podany w stopniach celsjusza.
  \item S3\_Temp - Odczyt z czujnika temperatury umieszczonego przy biurku nr 3, podany w stopniach celsjusza.
  \item S4\_Temp - Odczyt z czujnika temperatury umieszczonego przy biurku nr 4, podany w stopniach celsjusza.
  \item S1\_Light - Odczyt z czujnika światła umieszczonego przy biurku nr 1, podany w luxach.
  \item S2\_Light - Odczyt z czujnika światła umieszczonego przy biurku nr 2, podany w luxach.
  \item S3\_Light - Odczyt z czujnika światła umieszczonego przy biurku nr 3, podany w luxach.
  \item S4\_Light - Odczyt z czujnika światła umieszczonego przy biurku nr 4, podany w luxach.
  \item S1\_Sound - Odczyt z przetwornika analogowo-cyfrowego podłączonego do wyjścia wzmacniacza mikrofonu, wyrażony w woltach. Czujnik ten umieszczony jest przy biurku nr 1.
  \item S2\_Sound - Odczyt z przetwornika analogowo-cyfrowego podłączonego do wyjścia wzmacniacza mikrofonu, wyrażony w woltach. Czujnik ten umieszczony jest przy biurku nr 2.
  \item S3\_Sound - Odczyt z przetwornika analogowo-cyfrowego podłączonego do wyjścia wzmacniacza mikrofonu, wyrażony w woltach. Czujnik ten umieszczony jest przy biurku nr 3.
  \item S4\_Sound - Odczyt z przetwornika analogowo-cyfrowego podłączonego do wyjścia wzmacniacza mikrofonu, wyrażony w woltach. Czujnik ten umieszczony jest przy biurku nr 4.
  \item S5\_CO2 - Odczyt z detektora CO2, umieszczonego na środku pokoju, wyrażony w cząsteczkach na milion.
  \item S5\_CO2\_Slope - Zmiana stężenia CO2 w pomieszczeniu z detektora umieszczonego na środku pomieszczenia.
  \item S6\_PIR - Wartość prawda-fałsz wskazująca na wykrycie ruchu przez czujnik umieszczony nad drzwiami.
  \item S7\_PIR - Wartość prawda-fałsz wskazująca na wykrycie ruchu przez czujnik umieszczony na ścianie na przeciwko drzwi.
\end{itemize}

\chapter*{Projekt w RSES}


\begin{landscape}
\thispagestyle{mylandscape}
\chapter*{Wyniki}

W poniższej tabeli przedstawiono wyniki dla poszczególnych klasyfikatorów. Wyniki wyrażone są jako iloczyn dokładności i pokrycia zapisany jako procent.

\begin{table}[H]
  \resizebox{1.5\textwidth}{!}{
  \begin{tabular}{|c|c|c|c|c|c|c|c|c|c|c|c|c|c|c|c|c|c|c|c|c|c|c|}
  \hline
  \backslashbox{Klasyfikator}{Iteracja}  & 1 & 2 & 3 & 4 & 5 & 6 & 7 & 8 & 9 & 10 & 11 & 12 & 13 & 14 & 15 & 16 & 17 & 18 & 19 & 20 & Średnia \\ \hline
   Wyczerpujący  & 99.6\% & 99.4\% & 99.6\% & 99.5\% & 99.5\% & 99.6\% & 99.6\% & 99.5\% & 99.5\% & 99.7\% & 99.5\% & 99.4\% & 99.3\% & 99.6\% & 99.6\% & 99.4\% & 99.3\% & 99.7\% & 99.7\% & 99.6\% & 99.5\% \\ \hline
   Genetyczny    & 99.6\% & 99.5\% & 99.6\% & 99.6\% & 99.5\% & 99.6\% & 99.6\% & 99.5\% & 99.6\% & 99.7\% & 99.5\% & 99.5\% & 99.4\% & 99.6\% & 99.6\% & 99.4\% & 99.3\% & 99.7\% & 99.7\% & 99.6\% & 99.5\% \\ \hline
   Pokryciowy    & 83.4\% & 84.5\% & 84.1\% & 86.8\% & 83.9\% & 85.6\% & 83.8\% & 85.1\% & 57.8\% & 84.3\% & 84.5\% & 83.6\% & 85.3\% & 57.9\% & 83.6\% & 58.8\% & 81.6\% & 57.6\% & 83.5\% & 84.4\% & 79.0\% \\ \hline
   LEM2          & 92.5\% & 93.2\% & 92.8\% & 93.1\% & 92.1\% & 93.0\% & 93.2\% & 92.6\% & 92.3\% & 93.2\% & 93.0\% & 92.1\% & 92.2\% & 93.3\% & 92.3\% & 92.7\% & 92.5\% & 92.8\% & 92.4\% & 93.0\% & 92.72\% \\ \hline
  \end{tabular}}
  \end{table}

\end{landscape}

\chapter*{Wnioski}

Jak można zobaczyć po wynikach walidacji modelu, najlepszy wynik uzyskano na ścieżkach nr 2 oraz 5. Na podsatwie powyższych wyników, stwierdzić można że przygotowanie danych wykazuje poprawę względem danych nieprzygotowanych, a dyskretyzacja przy odpowiednio dobranej kolumnie danych pozwala uzyskać lepsze wyniki niż możnaby uzyskać przy pomocy normalizacji.

Zwrócić należy uwagę na fakt, że dyskretyzacja znacząco pogorszyła ocenę jakości w przypadku gdy brakujace wartości zastąpiono zerami. W pozostałych przypadkach natomiast wartości te nie odbiegają znacząco od siebie.

Ponadto na podstawie powyższych danych wysnuć należy wniosek, że odpowiedni dobór atrybutów poddanych dyskretyzacji pozwala uzyskać lepsze wyniki niż przeprowadzenie dyskretyzacji na całym zbiorze atrybutów.

Ostatecznie przygotowanie danych przed ich wykorzystaniem pozwala uzyskać znacząco lepsze rezultaty niż korzystanie z danych surowych. W świecie cyfryzacji, przygotowanie nawet dużych zbiorów danych nie stanowi zazwyczaj wyzwania. Przynajmniej nie stanowi wyzwania, które usprawiedliwia wykorzystanie surowego zbioru danych podczas analizy.

\end{document}